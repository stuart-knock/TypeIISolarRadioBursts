\documentclass[12pt,a4paper]{article} 
\pagestyle{plain}
\parindent=0.0cm

\usepackage[authoryear]{natbib} 

\textheight=25cm
\textwidth=16.3cm

\oddsidemargin=0.0cm
\evensidemargin=0.0cm

\marginparwidth=0.0cm
\marginparsep=0.0cm

\topmargin=0.0cm
\headheight=0.0cm
\headsep=0.0cm

\footskip=1.3cm

\include{commands}

\begin{document} 
 
\newlength{\temptextwidth} 

\begin{center}
{\Huge\bf A Code for Calculating Type II Dynamic Spectra}
 
 \vspace{1cm}
 
 \today
 
 \vspace{1cm}
\end{center}


%
\section{A Brief Description}
 The code {\it Dynamic\_Spectra.f90} calculates the  dynamic  spectra 
 predicted for a type II radio burst from the solar corona and
 interplanetary medium. Input values for the coronal, solar wind, and
 shock parameters are provided by way of the file {\it Dynamic\_Spectra.input}
 (for details see Subsection \ref{AppB:inputs} below, and the sample 
 input file included as Section \ref{AppB:inputfile}). The results
 of calculations are output as text files, and  include such things as
 a summary of the input parameters and run conditions, plasma and shock 
 parameters as a function of time, plasma and shock parameters as well as 
 fundamental and harmonic emission as a function of location, and 
 dynamic spectra.
 
 The data produced by {\it Dynamic\_Spectra.f90} is placed in a number of 
 different files (for details see Subsection \ref{AppB:outputs} below). 
 These data can then be plotted by means of two IDL scripts called  
 {\it Dynamic\_Spectra.pro} and {\it Helio\_cont.pro}.
 
 The three subsections below summarize: \ref{AppB:relatedfiles} the input
 file and plotting scripts;  \ref{AppB:inputs} the input variables; and 
 \ref{AppB:outputs} the output files related to {\it Dynamic\_Spectra.f90}.
%%%%%%%%%%%%%%%%%%%%%%%%%%%%%%%%%%%%%%%%%%%%%%%%%%%%%%%%%%%%%%%%%%%%% 
%%%%%%%%%%%%%%%%%%%%%%%%%%%%%%%%%%%%%%%%%%%%%%%%%%%%%%%%%%%%%%%%%%%%% 
%%%%%%%%%%%%%%%%%%%%%%%%%%%%%%%%%%%%%%%%%%%%%%%%%%%%%%%%%%%%%%%%%%%%% 
 

\section{Related Files:}\label{AppB:relatedfiles} 


 \setlength{\temptextwidth}{\textwidth}  
 \addtolength{\temptextwidth}{-1.0\parindent}
\noindent {\tt Dynamic\_Spectra.input}
  
 \parbox[t]{\temptextwidth}{Is the parameter input file. The inputs
                            include,  but are not limited to, the
                            output directory and file  information,
                            initial shock parameters, and the 1 AU 
                            and coronal normalizations of plasma and
                            magnetic field  parameters. The complete
                            list is described in Subsection 
                            \ref{AppB:inputs} below, and an example
                            input file is  included as Section
                            \ref{AppB:inputfile}.}

 \vspace{0.5cm}
 
 \setlength{\temptextwidth}{\textwidth}  
 \addtolength{\temptextwidth}{-1.0\parindent}
\noindent {\tt Dynamic\_Spectra.pro} 
 
 \parbox[t]{\temptextwidth}{Is an IDL script for plotting output from 
                            Dynamic\_Spectra.f90. It produces three
                            figures: the first is a five   panel
                            figure showing UCME, Radial\_Height,
                            theta, b, and U;  the second is a four
                            panel figure showing Ti, Te, N\_e, B1;
                            and  the third figure shows the dynamic
                            spectrum for a selected observer. This 
                            script is included as Section \ref{AppB:DSpro}}

 \vspace{0.5cm} 

 \setlength{\temptextwidth}{\textwidth}  
 \addtolength{\temptextwidth}{-1.0\parindent}
\noindent {\tt Helio\_cont.pro}
 
 \parbox[t]{\temptextwidth}{Is an IDL script for plotting output
                            from  Dynamic\_Spectra.f90. It produces a
                            10 panel figure of the plasma and 
                            magnetic field parameters as well as the
                            frequency integrated  fundamental and
                            harmonic fluxes as a function of
                            position. This  script is included as
                            Section \ref{AppB:HCpro}}
%%%%%%%%%%%%%%%%%%%%%%%%%%%%%%%%%%%%%%%%%%%%%%%%%%%%%%%%%%%%%%%%%%%%%
%%%%%%%%%%%%%%%%%%%%%%%%%%%%%%%%%%%%%%%%%%%%%%%%%%%%%%%%%%%%%%%%%%%%% 
%%%%%%%%%%%%%%%%%%%%%%%%%%%%%%%%%%%%%%%%%%%%%%%%%%%%%%%%%%%%%%%%%%%%% 
 
\newpage

\section{Inputs:}\label{AppB:inputs}
  {\it Dynamic\_Spectra.f90} gets its input from the file 
  {\it Dynamic\_Spectra.input}. What follows is a list of the
  input parameters  contained in {\it Dynamic\_Spectra.input}. The values
  listed below as  {\it DEFAULT} are used if the file {\it
  Dynamic\_Spectra.input} cannot be found.


  \setlength{\temptextwidth}{\textwidth}  
  \addtolength{\temptextwidth}{-1.0\parindent}
 \noindent {\tt Output\_dir} 
    
    \parbox[t]{\temptextwidth}{This is the path to the directory to which
                               data will be output.  It must already
                               exist, \ie the program will not create it. \\
                               $DEFAULT=$  {\tt 'data/spectra/'}}

  \vspace{0.5cm}

  \setlength{\temptextwidth}{\textwidth}   
  \addtolength{\temptextwidth}{-1.0\parindent}
 \noindent {\tt file\_ident} 
    
    \parbox[t]{\temptextwidth}{A distinguishing name to be appended to
                               the end of all of the output files for
                               the current run. \\ 
                               $DEFAULT=$ {\tt 'temp.dat'}}

  \vspace{0.5cm}

  \setlength{\temptextwidth}{\textwidth}  
  \addtolength{\temptextwidth}{-1.0\parindent} 
 \noindent {\tt Radial\_Height} 
    
    \parbox[t]{\temptextwidth}{The height in solar radii at which the
                               nose of the  global shock is positioned at
                               the start of  calculations. \\ 
                               ($214\approx$ 1 AU)         \\
                               $DEFAULT=$ {\tt 1.1}}
 
  \vspace{0.5cm}

  \setlength{\temptextwidth}{\textwidth}   
  \addtolength{\temptextwidth}{-1.0\parindent} 
 \noindent {\tt UCME} 
    
    \parbox[t]{\temptextwidth}{The initial radial speed of of the
                               global   shock in m s$^{-1}$. A reasonable
                               range is a few  hundred to a couple of
                               thousand km s$^{-1}$.
                               \\ $DEFAULT=$ {\tt 1500.0e3}}

  \vspace{0.5cm}

  \setlength{\temptextwidth}{\textwidth}   
  \addtolength{\temptextwidth}{-1.0\parindent}  
 \noindent {\tt CME\_accel} 
  
    \parbox[t]{\temptextwidth}{The deceleration imposed on the global 
                               shock speed  in m s$^{-2}$. A good guide
                               for selecting this number  is provided by 
                               \citet{gopalswamyetal01b}. \\
                               $DEFAULT=$ {\tt 5.9}}
      
  \vspace{0.5cm}

  \setlength{\temptextwidth}{\textwidth}  
  \addtolength{\temptextwidth}{-1.0\parindent}  
 \noindent {\tt expansion} 
  
    \parbox[t]{\temptextwidth}{A parameter to set the relative flow speed
                               across the  global shock front; ranging
                               between 0 and 1  corresponding to no
                               expansion (ie the way ripples are 
                               currently done) through to expansion
                               equivalent to its  radial speed.     \\
                               $DEFAULT=$ {\tt 0.0}}

  \vspace{0.5cm}

  \setlength{\temptextwidth}{\textwidth}  
  \addtolength{\temptextwidth}{-1.0\parindent} 
 \noindent {\tt scale} 
    
    \parbox[t]{\temptextwidth}{A parameter to set the size of the global
                               shock at 1 AU. A value of 1 gives a shock 
                               of ~1AU curvature at ~1AU, making this number
                               bigger makes the shock smaller at 1AU. \\
                               $DEFAULT=$ {\tt 1.0}}

  \vspace{0.5cm}
 

  \setlength{\temptextwidth}{\textwidth}   
  \addtolength{\temptextwidth}{-1.0\parindent} 
 \noindent {\tt scale\_exp} 
    
    \parbox[t]{\temptextwidth}{A parameter to define the rate of change 
                               of shock size with radial distance (see line 
                               882 of {\it Dynamic\_Spectra.f90}). It 
                               should generally be smaller for smaller 
                               1AU shocks so as to give a reasonable 
                               size at 1 Solar Radius. \\
                               $DEFAULT=$ {\tt 0.37}}

  \vspace{0.5cm}

  \setlength{\temptextwidth}{\textwidth}  
  \addtolength{\temptextwidth}{-1.0\parindent}   
 \noindent {\tt exp\_min} 
    
    \parbox[t]{\temptextwidth}{log$_{10}$ Hz of the lowest frequency in the 
                               calculated dynamic spectra. \\ 
                               $DEFAULT=$ {\tt 4.0}}

  \vspace{0.5cm}

\newpage

  \setlength{\temptextwidth}{\textwidth}   
  \addtolength{\temptextwidth}{-1.0\parindent} 
 \noindent {\tt exp\_range} 
   
    \parbox[t]{\temptextwidth}{log$_{10}$ Hz of the range of frequencies the
                               dynamic  spectra covers. \\
                               $DEFAULT=$ {\tt 5.0}}

  \vspace{0.5cm}

  \setlength{\temptextwidth}{\textwidth}  
  \addtolength{\temptextwidth}{-1.0\parindent} 
 \noindent {\tt l\_val} 
    
    \parbox[t]{\temptextwidth}{A multiplicative coefficient of the dynamic 
                               spectra frequency range. \\
                               $DEFAULT=$ {\tt 1.0}

                               Lower frequency = l\_val$\times 10^{exp\_min}$ \\ 
                               Upper frequency = l\_val$\times 10^{(exp\_min+exp\_range)}$}

  \vspace{0.5cm}

  \setlength{\temptextwidth}{\textwidth}   
  \addtolength{\temptextwidth}{-1.0\parindent} 
 \noindent {\tt Average\_kappa} 
    
    \parbox[t]{\temptextwidth}{Average value of KAPPA parameter of
                               distribution  function, 2-5 are reasonable
                               for solar wind. \\
                               $DEFAULT=$ {\tt 3.0}} 

  \vspace{0.5cm}

  \setlength{\temptextwidth}{\textwidth}  
  \addtolength{\temptextwidth}{-1.0\parindent} 
 \noindent {\tt Ne\_1AU} 
    
  \parbox[t]{\temptextwidth}{The 1 AU value for the electron number density 
                             (m$^{-3}$). The value at other radial distances 
                             is determined using a density model based on the
                             \citet{saitoetal77}, 
                             and \citet{rc98c} models. \\ 
                             $DEFAULT=$ {\tt 5.1e6}}

  \vspace{0.5cm}

  \setlength{\temptextwidth}{\textwidth}  
  \addtolength{\temptextwidth}{-1.0\parindent} 
 \noindent {\tt Ne\_1Rs} 
    
    \parbox[t]{\temptextwidth}{The 1 solar radius value for the electron number 
                               density (m$^{-3}$). The value at other radial 
                               distances 
                               is determined using a density model based on the
                               \citet{saitoetal77}, 
                               and \citet{rc98c} models. \\
                               $DEFAULT=$ {\tt 1.68e14}}

  \vspace{0.5cm}

  \setlength{\temptextwidth}{\textwidth}  
  \addtolength{\temptextwidth}{-1.0\parindent} 
 \noindent {\tt Te\_1AU} 
  
    \parbox[t]{\temptextwidth}{The 1 AU value for the electron temperature (K). 
                               The value at other radial distances 
                               is determined using a power-law electron temperature 
                               model from \citet{rc98c}. \\  
                               $DEFAULT=$ {\tt 2.4e5}}

  \vspace{0.5cm}

  \setlength{\temptextwidth}{\textwidth}  
  \addtolength{\temptextwidth}{-1.0\parindent} 
 \noindent {\tt Ti\_1AU} 
  
    \parbox[t]{\temptextwidth}{The 1 AU value for the ion temperature (K). The 
                               value at other radial distances is determined 
                               using the functional form used for $T_e$, but 
                               with an exponent which produces a typical coronal 
                               value for a nominal 1 AU value. \\ 
                               $DEFAULT=$ {\tt 1.7e5}}

  \vspace{0.5cm}

  \setlength{\temptextwidth}{\textwidth}  
  \addtolength{\temptextwidth}{-1.0\parindent} 
 \noindent {\tt Bo} 
    \parbox[t]{\temptextwidth}{The magnetic field strength (T) at 1 solar radius. 
                               The  value at other radial
                               distances is determined  using the Parker
                               model for equatorial magnetic field
                               strength, the default value provides typical 
                               value at 1 AU) \\ 
                               $DEFAULT=$ {\tt 1.89743128e-4}}

  \vspace{0.5cm}

  \setlength{\temptextwidth}{\textwidth}  
  \addtolength{\temptextwidth}{-1.0\parindent} 
 \noindent {\tt Vsw\_1AU} 
  
    \parbox[t]{\temptextwidth}{The solar wind speed at 1 AU (m s${-1}$). The value at
                               other radial distances  is determined
                               using a solar wind flow speed model from  
                               \citet{rc98c}.\\ 
                               $DEFAULT=$ {\tt 4.0e5}}

  \vspace{0.5cm}

  \setlength{\temptextwidth}{\textwidth}  
  \addtolength{\temptextwidth}{-1.0\parindent} 
 \noindent {\tt Xo1} 
  
    \parbox[t]{\temptextwidth}{The X coordinate (heliocentric) of observer 1 (m). \\ 
                               $DEFAULT=$ {\tt 1.49599e11}}

  \vspace{0.5cm}

  \setlength{\temptextwidth}{\textwidth}  
  \addtolength{\temptextwidth}{-1.0\parindent} 
 \noindent {\tt Yo1} 
    
    \parbox[t]{\temptextwidth}{The Y coordinate (heliocentric) of observer 1 (m). \\ 
                               $DEFAULT=$ {\tt 1.00e9}}

  \vspace{0.5cm}

  \setlength{\temptextwidth}{\textwidth}  
  \addtolength{\temptextwidth}{-1.0\parindent} 
 \noindent {\tt Xo2} 
    
    \parbox[t]{\temptextwidth}{The X coordinate (heliocentric) of observer 2 (m). \\ 
                               $DEFAULT=$ {\tt 1.00e11}}

  \vspace{0.5cm}

  \setlength{\temptextwidth}{\textwidth}   
  \addtolength{\temptextwidth}{-1.0\parindent} 
 \noindent {\tt Yo2} 
    
    \parbox[t]{\temptextwidth}{The Y coordinate (heliocentric) of observer 2 (m). \\ 
                               $DEFAULT=$ {\tt 1.00e11}}

  \vspace{0.5cm}

  \setlength{\temptextwidth}{\textwidth}  
  \addtolength{\temptextwidth}{-1.0\parindent} 
 \noindent {\tt Xo3} 
    
    \parbox[t]{\temptextwidth}{The X coordinate (heliocentric) of observer 3 (m). \\ 
                               $DEFAULT=$ {\tt 1.00e11}}

  \vspace{0.5cm}

  \setlength{\temptextwidth}{\textwidth}  
  \addtolength{\temptextwidth}{-1.0\parindent} 
 \noindent {\tt Yo3} 
    
    \parbox[t]{\temptextwidth}{The Y coordinate (heliocentric) of observer 3 (m). \\ 
                               $DEFAULT=$ {\tt -1.00e11}}

  \vspace{0.5cm}

  \setlength{\temptextwidth}{\textwidth}  
  \addtolength{\temptextwidth}{-1.0\parindent} 
 \noindent {\tt time\_resolution} 
    
    \parbox[t]{\temptextwidth}{The time resolution of dynamic spectra in seconds. \\ 
                               $DEFAULT=$ {\tt 10}}

  \vspace{0.5cm}

  \setlength{\temptextwidth}{\textwidth}  
  \addtolength{\temptextwidth}{-1.0\parindent} 
 \noindent {\tt Maxtheta} 
    
    \parbox[t]{\temptextwidth}{The upper range in radians for the array
                               of theta  values used in the calculation 
                               of ripples as a function of time -- 
                               output by the  code. \\ 
                               $DEFAULT=$ {\tt 6.283185307}}

  \vspace{0.5cm}

  \setlength{\temptextwidth}{\textwidth}  
  \addtolength{\temptextwidth}{-1.0\parindent} 
 \noindent {\tt Mintheta} 
   
    \parbox[t]{\temptextwidth}{The lower range in radians for the array
                               of theta  values used in the calculation 
                               of ripples as a function of time -- 
                               output by the  code. \\ 
                               $DEFAULT=$ {\tt 3.141592654}}

  \vspace{0.5cm}

  \setlength{\temptextwidth}{\textwidth}  
  \addtolength{\temptextwidth}{-1.0\parindent} 
 \noindent {\tt MaxU} 
    
    \parbox[t]{\temptextwidth}{The upper range in m s$^{-1}$ for the array 
                               of flow speed values used in the calculation 
                               of ripples as a function of time -- 
                               output by the  code. \\ 
                               $DEFAULT=$ {\tt 100.0e4}}

  \vspace{0.5cm}

  \setlength{\temptextwidth}{\textwidth}  
  \addtolength{\temptextwidth}{-1.0\parindent} 
 \noindent {\tt MinU} 
    
    \parbox[t]{\temptextwidth}{The lower range in m s$^{-1}$ for the array 
                               of flow speed values used in the calculation 
                               of ripples as a function of time -- 
                               output by the  code. \\ 
                               $DEFAULT=$ {\tt 0.0e1}}

  \vspace{0.5cm}

  \setlength{\temptextwidth}{\textwidth}  
  \addtolength{\temptextwidth}{-1.0\parindent} 
 \noindent {\tt Maxb} 
    
    \parbox[t]{\temptextwidth}{The upper range in log$_{10}$[m$^{-1}$] for 
                               the array of ripple curvature values used in 
                               the calculation of ripples as a function of 
                               time -- output by the  code. \\ 
                               $DEFAULT=$ {\tt -7.0}}

  \vspace{0.5cm}

  \setlength{\temptextwidth}{\textwidth}   
  \addtolength{\temptextwidth}{-1.0\parindent} 
 \noindent {\tt Minb} 
    
    \parbox[t]{\temptextwidth}{The lower range in log$_{10}$[m$^{-1}$] for 
                               the array of ripple curvature values used in 
                               the calculation of ripples as a function of 
                               time -- output by the  code. \\ 
                               $DEFAULT=$ {\tt -10.0}}

  \vspace{0.5cm}

  \setlength{\temptextwidth}{\textwidth}  
  \addtolength{\temptextwidth}{-1.0\parindent} 
 \noindent {\tt MaxB1} 
    
    \parbox[t]{\temptextwidth}{The upper range in log$_{10}$[T] for the 
                               array of magnetic field strength values used in 
                               the calculation of ripples as a function of 
                               time -- output by the  code. \\ 
                               $DEFAULT=$ {\tt -3.0}}

  \vspace{0.5cm}

  \setlength{\temptextwidth}{\textwidth}  
  \addtolength{\temptextwidth}{-1.0\parindent} 
 \noindent {\tt MinB1} 
    
    \parbox[t]{\temptextwidth}{The lower range in log$_{10}$[T] for the  
                               array of magnetic field strength values used in 
                               the calculation of ripples as a function of 
                               time -- output by the  code. \\ 
                               $DEFAULT=$ {\tt -9.0}}

  \vspace{0.5cm}

  \setlength{\temptextwidth}{\textwidth} 
  \addtolength{\temptextwidth}{-1.0\parindent} 
 \noindent {\tt MaxNe} 
    
    \parbox[t]{\temptextwidth}{The upper range in log$_{10}$[m$^{-3}$] for the 
                               array of electron number density values used in 
                               the calculation of ripples as a function of 
                               time -- output by the  code. \\ 
                               $DEFAULT=$ {\tt 15.0}}

  \vspace{0.5cm}

  \setlength{\temptextwidth}{\textwidth} 
  \addtolength{\temptextwidth}{-1.0\parindent} 
 \noindent {\tt MinNe} 
    
    \parbox[t]{\temptextwidth}{The lower range in log$_{10}$[m$^{-3}$] for the 
                               array of electron number density values used in 
                               the calculation of ripples as a function of 
                               time -- output by the  code. \\ 
                               $DEFAULT=$ {\tt 6.0}}

  \vspace{0.5cm}

  \setlength{\temptextwidth}{\textwidth} 
  \addtolength{\temptextwidth}{-1.0\parindent} 
 \noindent {\tt MaxTe} 
   
    \parbox[t]{\temptextwidth}{The upper range in K for the array of electron 
                               temperature used in the calculation of ripples 
                               as a function of time -- output by the  code. \\ 
                               $DEFAULT=$ {\tt 2.0e6}} 

  \vspace{0.5cm}

  \setlength{\temptextwidth}{\textwidth}  
  \addtolength{\temptextwidth}{-1.0\parindent} 
 \noindent {\tt MinTe} 
    
    \parbox[t]{\temptextwidth}{The lower range in K for the array of electron 
                               temperature used in the calculation of ripples 
                               as a function of time -- output by the  code. \\ 
                               $DEFAULT=$ {\tt 4.0e4}}

  \vspace{0.5cm}

  \setlength{\temptextwidth}{\textwidth} 
  \addtolength{\temptextwidth}{-1.0\parindent} 
 \noindent {\tt MaxTi} 
    
    \parbox[t]{\temptextwidth}{The upper range in K for the array of ion 
                               temperature used in the calculation of ripples 
                               as a function of time -- output by the  code. \\ 
                               $DEFAULT=$ {\tt 3.0e6}} 

  \vspace{0.5cm}

  \setlength{\temptextwidth}{\textwidth} 
  \addtolength{\temptextwidth}{-1.0\parindent} 
 \noindent {\tt MinTi} 
    
    \parbox[t]{\temptextwidth}{The lower range in K for the array of ion 
                               temperature used in the calculation of ripples 
                               as a function of time -- output by the  code. \\ 
                               $DEFAULT=$ {\tt 4.0e4}}

  \vspace{0.5cm}

  \setlength{\temptextwidth}{\textwidth}  
  \addtolength{\temptextwidth}{-1.0\parindent} 
 \noindent {\tt nos} 
    
    \parbox[t]{\temptextwidth}{The number of plasma structures imposed on 
                               the model corona and solar wind. Primary use 
                               is so that the code knows how many to try and 
                               read in. \\ 
                               $DEFAULT=$ {\tt 0}}

 \vspace{0.5cm}       

\noindent For examples of structure parameters see Dynamic\_Spectra.input and the
Structures subroutine in Dynamic\_Spectra.f90.
%%%%%%%%%%%%%%%%%%%%%%%%%%%%%%%%%%%%%%%%%%%%%%%%%%%%%%%%%%%%%%%%%%%%% 
%%%%%%%%%%%%%%%%%%%%%%%%%%%%%%%%%%%%%%%%%%%%%%%%%%%%%%%%%%%%%%%%%%%%% 
%%%%%%%%%%%%%%%%%%%%%%%%%%%%%%%%%%%%%%%%%%%%%%%%%%%%%%%%%%%%%%%%%%%%% 

\newpage

\section{Outputs:}\label{AppB:outputs}   
The code replaces the $\ast$ in what follows by file\_ident, see page \pageref{AppB:inputs}. 

  \setlength{\temptextwidth}{\textwidth}  
  \addtolength{\temptextwidth}{-1.0\parindent}
 \noindent {\tt Source\_Loc\_$\ast$} 
    
    \parbox[t]{\temptextwidth}{Five numbers output for every ripple calculated, consisting  of:
       
                               'X' position defined relative to the time step global shock  coordinates. 

                               'Y' corresponding to X 

                               'F' Fundamental flux at observer generated  by that source. 
                               
                               'H' Harmonic flux at observer generated by that source.

                               '$f_p$' The average plasma frequency of the source.}

  
   
  \setlength{\temptextwidth}{\textwidth}  
  \addtolength{\temptextwidth}{-1.0\parindent}
 \noindent {\tt Source\_Num\_$\ast$} 
   
   \parbox[t]{\temptextwidth}{The number of sources/ripples calculated per time step 
                               in order to pack the global shock. Used to work out the 
                               the number of sources in a given number of time steps.}

    \vspace{0.5cm}
   
  \setlength{\temptextwidth}{\textwidth}  
  \addtolength{\temptextwidth}{-1.0\parindent}
 \noindent {\tt CME\_b\_$\ast$} 
    
    \parbox[t]{\temptextwidth}{The curvature parameter for the global shock at successive time steps.}  

    \vspace{0.5cm}
   
  \setlength{\temptextwidth}{\textwidth}   
  \addtolength{\temptextwidth}{-1.0\parindent}
 \noindent {\tt DS{[1-3][F$|$H]}\_$\ast$} 
    
    \parbox[t]{\temptextwidth}{An array of the flux in log$_{10}$ W m$^{-2}$ Hz$^{-1}$ sr$^{-1}$ as 
                               a function of frequency and time. [1-3] and [F$|$H] refer to the six 
                               different dynamic spectra for the 3 different observers as separate 
                               fundamental and harmonic components, \eg DS1F\_$\ast$ is the fundamental 
                               component of the dynamic spectrum seen by observer 1.}
  
  \vspace{0.5cm}
   
   \setlength{\temptextwidth}{\textwidth}  
  \addtolength{\temptextwidth}{-1.0\parindent}
 \noindent {\tt LOG\_$\ast$} 
    
    \parbox[t]{\temptextwidth}{General information about a run:
    Includes start time, array sizes, resolution, input parameter 
    plot ranges, and finish time and reason for finish.}  

   \vspace{0.5cm}
   
   \setlength{\temptextwidth}{\textwidth}   
  \addtolength{\temptextwidth}{-1.0\parindent}
 \noindent {\tt parameters\_$\ast$} 
    
    \parbox[t]{\temptextwidth}{Consists of: input parameters followed by the time of time steps in 
                               seconds and the corresponding solar wind speed of the 
                               last ripple calculation. Can be useful in determining the time in 
                               seconds corresponding to a number of time steps.}

    \vspace{0.5cm}
   
  \setlength{\temptextwidth}{\textwidth}  
  \addtolength{\temptextwidth}{-1.0\parindent}
 \noindent {\tt scaling\_$\ast$} 
    
    \parbox[t]{\temptextwidth}{Values used for conversion/stretching of values output on a 
                               time step basis into an actual time spread...}
  
   \vspace{0.5cm}
   
   \setlength{\temptextwidth}{\textwidth}   
  \addtolength{\temptextwidth}{-1.0\parindent}
 \noindent {\tt Radial\_Height\_$\ast$} 
   
    \parbox[t]{\temptextwidth}{The height above the Sun of the leading edge of the global shock 
                               at successive time steps.}
  
   \vspace{0.5cm}
   
  \setlength{\temptextwidth}{\textwidth}  
  \addtolength{\temptextwidth}{-1.0\parindent}
 \noindent {\tt UCME\_$\ast$} 
    
    \parbox[t]{\temptextwidth}{The radial speed m s$^{-1}$ of the global shock at successive time steps.}
 
  \vspace{0.5cm}
   
  \setlength{\temptextwidth}{\textwidth}  
  \addtolength{\temptextwidth}{-1.0\parindent}
 \noindent {\tt theta\_$\ast$} 
    
    \parbox[t]{\temptextwidth}{Ripples at subsequent time steps binned as a function of the angle 
                               between the negative of the bulk plasma flow and the upstream magnetic 
                               field. (bin ranges are given in LOG\_$\ast$)}
  
  \vspace{0.5cm}
   
  \setlength{\temptextwidth}{\textwidth}  
  \addtolength{\temptextwidth}{-1.0\parindent}
 \noindent {\tt U\_$\ast$} 
    
    \parbox[t]{\temptextwidth}{Ripples at subsequent time steps binned as a function of the flow 
                               speed.}
  
  \vspace{0.5cm}
   
  \setlength{\temptextwidth}{\textwidth}  
  \addtolength{\temptextwidth}{-1.0\parindent}
 \noindent {\tt b\_$\ast$} 
    \parbox[t]{\temptextwidth}{Ripples at subsequent time steps binned as a function of the 
                               ripple curvature, and thus size.}
  
  \vspace{0.5cm}
   
  \setlength{\temptextwidth}{\textwidth}   
  \addtolength{\temptextwidth}{-1.0\parindent}
 \noindent {\tt B1\_$\ast$}
    
    \parbox[t]{\temptextwidth}{Ripples at subsequent time steps binned as a function of the 
                               upstream magnetic field strength.}

  \vspace{0.5cm}
   
  \setlength{\temptextwidth}{\textwidth}  
  \addtolength{\temptextwidth}{-1.0\parindent}
 \noindent {\tt Ne\_$\ast$} 
   
    \parbox[t]{\temptextwidth}{Ripples at subsequent time steps binned as a function of the 
                               electron number density.}
  
  \vspace{0.5cm}
 
  \setlength{\temptextwidth}{\textwidth}  
  \addtolength{\temptextwidth}{-1.0\parindent}
 \noindent {\tt Te\_$\ast$} 
    
    \parbox[t]{\temptextwidth}{Ripples at subsequent time steps binned as a function of the 
                               electron temperature.}
 
  \vspace{0.5cm}
   
  \setlength{\temptextwidth}{\textwidth}  
  \addtolength{\temptextwidth}{-1.0\parindent}
 \noindent {\tt Ti\_$\ast$} 
    
    \parbox[t]{\temptextwidth}{Ripples at subsequent time steps binned as a function of the ion 
                               temperature.}



%%%
\begin{thebibliography}{}

\bibitem[{\it Gopalswamy et al.}(2001b)]{gopalswamyetal01b}
 Gopalswamy, N., A. Lara, S. Yashiro, M.L. Kaiser, 
and R.A. Howard, 
Predicting the 1-AU arrival times of coronal mass ejections, 
{\it Journal of Geophysical Research} {\it 106}, 29207, 2001.


\bibitem[{\it Robinson and Cairns}(1998c)]{rc98c}
 Robinson, P.A., and I.H. Cairns, 
Fundamental and Harmonic Emission in Type III Solar Radio Bursts 
III. Heliocentric Variations of Interplanetary Beam and Source 
Parameters, 
{\it Solar Physics}, {\it 181}, 429, 1998c.


\bibitem[{\it Saito et al.}(1977)]{saitoetal77}
Saito, Kuniji, Arthur I. Poland, and Richard H. Munro, 
A Study of the Background Corona Near Solar Minimum, 
{\it Solar Physics}, {\it 57}, 121, 1977.
%%%
\end{thebibliography}
%%%

\end{document}
